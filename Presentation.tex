\documentclass{beamer}
 

\usepackage[ngerman]{babel}
 
% Sourcecode with colors
\usepackage{color}
\definecolor{dkgreen}{rgb}{0,0.6,0}
\definecolor{gray}{rgb}{0.5,0.5,0.5}
\definecolor{mauve}{rgb}{0.58,0,0.82}
 
% Use sourcecode package
\usepackage{listings}
\lstset{numbers=left,
	numberstyle=\tiny,
	numbersep=5pt,
	breaklines=true,
	showstringspaces=false,
	frame=l ,
	xleftmargin=15pt,
	xrightmargin=15pt,
	basicstyle=\ttfamily\scriptsize,
	stepnumber=1,
	keywordstyle=\color{blue},          % keyword style
  	commentstyle=\color{dkgreen},       % comment style
  	stringstyle=\color{mauve}         % string literal style
}
% Programming language
\lstset{language=JAVA}
 
\begin{document}

% Our title slide
\title{NXT Hackathon}
\subtitle{Learning about robots with lego}
\author
{Patrick Nagel \and Tim Metzler }
\institute[HBRS]
{
  Autonomous Systems\\
  Bonn-Rhein-Sieg University\\
  of Applied Science
}
\date{\today} % (optional)
\subject{Computer Science}
\frame{\titlepage} 
 
 
 \frame{\frametitle{Table of contents}
\begin{itemize}
\item Line Follower - Problem
\item Line Follower - P Controller
\item Line Follower - Sourcecode
\item Line Follower - Demonstration 
\item D
\end{itemize} 
}

\frame{\frametitle{Line Follower - Problem}
\textbf{Problem}: Follow a black line with the NXT using only light sensors.\\
\vspace{1em}
\textbf{Solution}: Use two light sensors at each side of the line. Each sensor should see the bright surface next to the line.\\
Light sensor values are read every 10ms. If the right sensor is dark we go to the right. If the left sensor is dark we got to the left.\\
If both sensors are dark we have passed the finish line.

}

\frame{\frametitle{Line Follower - P Controller}
\textbf{P Controller}: Measure the error from the sensors. Adjust the motor speeds by adding / subtracting a correction value. \\
Correction value = $K_{p}$ * error, error $\in \{-1,0,1\}$\\


}

% A slide with a bit of source code
\begin{frame}[fragile] %fragile ist sehr wichtig!
\frametitle{LineFollower - Sourcecode}
\begin{lstlisting}
public class FollowTheLine {

	public static final int TOP_SPEED = 200;

	// The constant for the proportional controller
	private int kp;
	// The correction value (how much we slow down / speed up the motors)
	private int turn;
	// The error value (-1, 0, 1)
	private int error;

	// Calibration values
	private int darkRight;
	private int darkLeft;

	private int brightRight;
	private int brightLeft;

	private int triggerRight;
	private int triggerLeft;

	private LightSensor lsRight;
	private LightSensor lsLeft;

\end{lstlisting}
\end{frame}

\begin{frame}[fragile]
\frametitle{LineFollower - Sourcecode continued}
\begin{lstlisting}[firstnumber = 24]

	public FollowTheLine() {
		// The value of 120 is selected by trial and error
		kp = 120; 

		// Initialize the light sensors
		lsRight = new LightSensor(SensorPort.S1);
		lsLeft = new LightSensor(SensorPort.S2);

		// Calibrate the two light sensors
		calibrateSensors();

		// Start to follow the course
		followTheCourse();
	}

\end{lstlisting}
\end{frame}

\begin{frame}[fragile]
\frametitle{LineFollower - Sourcecode continued}
\begin{lstlisting}[firstnumber = 39]

	/**
	 * Start following the course
	 */
	public void followTheCourse() {

		Motor.A.setSpeed(TOP_SPEED);
		Motor.B.setSpeed(TOP_SPEED);

		// Smoother acceleration
		Motor.A.setAcceleration(2000);
		Motor.B.setAcceleration(2000);

		// Start moving
		Motor.A.forward();
		Motor.B.forward();

		int lightValueRight;
		int lightValueLeft;

		while (!Button.ESCAPE.isDown()) {
			error = 0;

			// Small delay
			Delay.msDelay(100); 
			
\end{lstlisting}
\end{frame}

\begin{frame}[fragile]
\frametitle{LineFollower - Sourcecode continued}
\begin{lstlisting}[firstnumber = 64]

		// Get the current light values of both sensors
		lightValueRight = lsRight.getLightValue();
		lightValueLeft = lsLeft.getLightValue();

		// If both sensors see dark, we reached the finish line
		if (lightValueLeft < triggerLeft && lightValueRight < triggerRight) {
			Motor.A.stop();
			Motor.B.stop();
			return;
		}
			
		if (lightValueRight < triggerRight) {
			error = -1;
		}

		if (lightValueLeft < triggerLeft) {
			error = 1;
		}
		
		\end{lstlisting}
\end{frame}

\begin{frame}[fragile]
\frametitle{LineFollower - Sourcecode continued}
\begin{lstlisting}[firstnumber = 83]

		// P-Controller
		turn = kp * error;

		// Adjust the speed
		Motor.A.setSpeed(TOP_SPEED + turn);
		Motor.B.setSpeed(TOP_SPEED - turn);

		// Handle the turn direction
		if (error < 0) {
			Motor.A.backward();
			Motor.B.forward();
		}

		if (error > 0) {
			Motor.A.forward();
			Motor.B.backward();
		}

		if (error == 0) {
			Motor.A.forward();
			Motor.B.forward();
		}
	}

\end{lstlisting}
\end{frame}

\begin{frame}{}
  \centering \Large
  \emph{\vspace{1em}Live demonstration of the Line Follower}
\end{frame}


% Our final slide
\begin{frame}{}
  \centering \Large
  \emph{\vspace{1em}Thank you!\\Any questions?}
\end{frame}
 
\end{document}
