\documentclass{beamer}
 

\usepackage[ngerman]{babel}
 
% Sourcecode with colors
\usepackage{color}
\definecolor{dkgreen}{rgb}{0,0.6,0}
\definecolor{gray}{rgb}{0.5,0.5,0.5}
\definecolor{mauve}{rgb}{0.58,0,0.82}
 
% Use sourcecode package
\usepackage{listings}
\lstset{numbers=left,
	numberstyle=\tiny,
	numbersep=5pt,
	breaklines=true,
	showstringspaces=false,
	frame=l ,
	xleftmargin=15pt,
	xrightmargin=15pt,
	basicstyle=\ttfamily\scriptsize,
	stepnumber=1,
	keywordstyle=\color{blue},          % keyword style
  	commentstyle=\color{dkgreen},       % comment style
  	stringstyle=\color{mauve}         % string literal style
}
% Programming language
\lstset{language=JAVA}
 
\begin{document}

% Our title slide
\title{NXT Hackathon}
\subtitle{Learning about robots with lego}
\author
{Patrick Nagel \and Tim Metzler }
\institute[HBRS]
{
  Autonomous Systems\\
  Bonn-Rhein-Sieg University\\
  of Applied Science
}
\date{\today} % (optional)
\subject{Computer Science}
\frame{\titlepage} 
 
 
 \frame{\frametitle{Table of contents}
\begin{itemize}
\item Line Follower - Problem
\item Line Follower - P Controller
\item Line Follower - Sourcecode
\item C 
\item D
\end{itemize} 
}

\frame{\frametitle{Line Follower - Problem}
\textbf{Problem}: Follow a black line with the NXT using only light sensors.\\
\vspace{1em}
\textbf{Solution}: Use two light sensors at each side of the line. Each sensor should see the bright surface next to the line.\\
Light sensor values are read every 10ms. If the right sensor is dark we go to the right. If the left sensor is dark we got to the left.\\
If both sensors are dark we have passed the finish line.

}

\frame{\frametitle{Line Follower - P Controller}
\textbf{P Controller}: Measure the error from the sensors. Adjust the motor speeds by adding / subtracting a correction value. \\
Correction value = $K_{p}$ * error, error $\in \{-1,0,1\}$\\


}

% A slide with a bit of source code
\begin{frame}[fragile] %fragile ist sehr wichtig!
\frametitle{LineFollower - Sourcecode}
\begin{lstlisting}
public class TestClass {
	public static void main(String[] args) {
		System.out.println(``Hello World'');
	}
}
\end{lstlisting}
\end{frame}

% Our final slide
\begin{frame}{}
  \centering \Large
  \emph{\vspace{1em}Thank you!\\Any questions?}
\end{frame}
 
\end{document}
